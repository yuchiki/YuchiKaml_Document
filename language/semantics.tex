
\subsection{Semantics}

Then we define the semantics of expressions.
We assume that the behaviour of built-in functions are given.
That is, built-in functions are assumed deterministic.

\subsubsection{Value}
\emph{Values} of expressions are listed as below:

\begin{align*}
    v (\text{value}) &\defeq n \mid b \mid  s \mid  \closure \mid  \builtinFunction\\
    \valuation  (\text{valuation}) &\in \variables \not\to\values \\
    \builtinFunction{} (\text{built-in function})&\in \History \times\values\not\to\values\\
    \closure (\text{closure}) &\defeq (x,e,\valuation{})\\
    h (\text{history}) &\defeq \seq{(f_1, v_1); \cdots; (f_n, v_n)}
\end{align*}



Here $\variables$ is the set of the variables and $\values$ is the set of the values.

\begin{note}
We assume that all the built-in functions return the same value for the same input, if it does not diverge.
This is enough to see weather different two ways of execution of a program is equivalent or not ,given a situation.
\end{note}

\subsubsection{Evaluation Relation}
\label{sec: Evaluation Realation}

We have to define the behaviour of evaluation process of expressions clearly.
To this end, we define the \emph{evaluation} process of expression by a big-step semantics shown below.

An \emph{evaluation relation} is a four-term relation of the form $\evalRel{\valuation}{h}{e}{h'}{v}$.

The evaluation rules of YuchiKaml are shown below:

\infrule[E-Var]
{}
{\evalRel{\valuation}{h}{x}{h}{\valuation(x)}}

\infrule[E-App]
{\evalRel{\valuation}{h}{e_1}{h'}{(x,e_1',\valuation')} \andalso
{\evalRel{\valuation}{h'}{e_2}{h''}{v_2}}}
{\evalRel{\valuation}{h}{e_1\ e_2}{h'''}{v_1'}}

\infrule[E-Not]
{\evalRel{\valuation}{h}{e}{h'}{b}}
{\evalRel{\valuation}{h}{!e}{h'}{\interp{!}b}}

\infrule[E-Mul]
{\evalRel{\valuation}{h}{e_1}{h'}{n_1} \andalso
\evalRel{\valuation}{h'}{e_2}{h''}{n_2}}
{\evalRel{\valuation}{h}{e_1 * e_2}{h''}{n_1 \interp{*}n_2}}

\infrule[E-Div]
{\evalRel{\valuation}{h}{e_1}{h'}{n_1} \andalso
\evalRel{\valuation}{h'}{e_2}{h''}{n_2}}
{\evalRel{\valuation}{h}{e_1 / e_2}{h''}{n_1 \interp{/}n_2}}


\infrule[E-Plus]
{\evalRel{\valuation}{h}{e_1}{h'}{n_1} \andalso
\evalRel{\valuation}{h'}{e_2}{h''}{n_2}}
{\evalRel{\valuation}{h}{e_1 + e_2}{h''}{n_1 \interp{+}n_2}}

\infrule[E-Minus]
{\evalRel{\valuation}{h}{e_1}{h'}{n_1} \andalso
\evalRel{\valuation}{h'}{e_2}{h''}{n_2}}
{\evalRel{\valuation}{h}{e_1 - e_2}{h''}{n_1 \interp{-}n_2}}

\infrule[E-Leq]
{\evalRel{\valuation}{h}{e_1}{h'}{n_1} \andalso
\evalRel{\valuation}{h'}{e_2}{h''}{n_2}}
{\evalRel{\valuation}{h}{e_1 \leq e_2}{h''}{n_1 \interp{\leq}n_2}}

\infrule[E-Lt]
{\evalRel{\valuation}{h}{e_1}{h'}{n_1} \andalso
\evalRel{\valuation}{h'}{e_2}{h''}{n_2}}
{\evalRel{\valuation}{h}{e_1 < e_2}{h''}{n_1 \interp{<}n_2}}

\infrule[E-Geq]
{\evalRel{\valuation}{h}{e_1}{h'}{n_1} \andalso
\evalRel{\valuation}{h'}{e_2}{h''}{n_2}}
{\evalRel{\valuation}{h}{e_1 \geq e_2}{h''}{n_1 \interp{\geq}n_2}}

\infrule[E-Gt]
{\evalRel{\valuation}{h}{e_1}{h'}{n_1} \andalso
\evalRel{\valuation}{h'}{e_2}{h''}{n_2}}
{\evalRel{\valuation}{h}{e_1 > e_2}{h''}{n_1 \interp{>}n_2}}

\infrule[E-Eq]
{\evalRel{\valuation}{h}{e_1}{h'}{v_1} \andalso
\evalRel{\valuation}{h'}{e_2}{h''}{v_2}}
{\evalRel{\valuation}{h}{e_1 = e_2}{h''}{v_1 \interp{=}v_2}}

\infrule[E-Neq]
{\evalRel{\valuation}{h}{e_1}{h'}{v_1} \andalso
\evalRel{\valuation}{h'}{e_2}{h''}{v_2}}
{\evalRel{\valuation}{h}{e_1 \not= e_2}{h''}{v_1 \interp{\not=}v_2}}

\infrule[E-And]
{\evalRel{\valuation}{h}{e_1}{h'}{b_1} \andalso
\evalRel{\valuation}{h'}{e_2}{h''}{b_2}}
{\evalRel{\valuation}{h}{e_1 \logAnd{} e_2}{h''}{v_1 \interp{\logAnd}v_2}}

\infrule[E-Or]
{\evalRel{\valuation}{h}{e_1}{h'}{b_1} \andalso
\evalRel{\valuation}{h'}{e_2}{h''}{b_2}}
{\evalRel{\valuation}{h}{e_1 \logOr{} e_2}{h''}{v_1 \interp{\logOr} v_2}}

\infrule[E-If-True]
{\evalRel{\valuation}{h}{e_1}{h'}{\trueToken} \andalso
\evalRel{\valuation}{h'}{e_2}{h''}{v_2}}
{\evalRel{\valuation}{h}{\ifStructure{e_1}{e_2}{e_3}}{h''}{v_2}}

\infrule[E-If-False]
{\evalRel{\valuation}{h}{e_1}{h'}{\falseToken} \andalso
\evalRel{\valuation}{h'}{e_3}{h''}{v_3}}
{\evalRel{\valuation}{h}{\ifStructure{e_1}{e_2}{e_3}}{h''}{v_3}}

\infrule[E-Let]
{
    \evalRel{\valuation}{h}{e_1}{h'}{v_1} \andalso
    \evalRel{\valuation\cup\set{x\mapsto v_1}}{h'}{e_2}{h''}{v_2}
}{
    \evalRel{\valuation}{h}{\letStructure{x}{e_1}{e_2}}{h''}{v_2}
}

\infrule[E-LetRec]
{\evalRel{\mu X. \valuation\cup\set{f \mapsto (x, e_1, X)}}{h}{e_2}{h'}{v_2}}
{\evalRel{\valuation}{h}{\letRecStructure{f}{x}{e_1}{e_2}}{h'}{v_2}}

\infrule[E-AppBuiltIn]
{\evalRel{\valuation}{h}{e_1}{h'}{\builtinFunction} \andalso
\evalRel{\valuation}{h'}{e_2}{h''}{v_2}}
{\evalRel{\valuation}{h}{e_1\ e_2}{\seq{h'', (\builtinFunction, v_2)}}{(\builtinFunction h'' v_2)}}

\infrule[E-Abs]
{}
{\evalRel{\valuation}{h}{\abs{x}{e}}{h}{(x,e,\valuation{})}}

We write $\interp{e}$ for $(h, v)$ such that $\evalRel{\emptyset}{\seq{}}{e}{h}{v}$.

\subsubsection{Valid Execution}

The evaluation process shown above is just one of the ways of possible evaluations.
Real Implementation of YuchiKaml may be done in a more optimized way, or in av completely different way which gives the result equivalent to that of the process in Section \ref{sec: Evaluation Realation}.
We define the valid ways of executions using the evaluation relation.

\begin{definition}[valid execution]
    Assume that the behaviour of built-in functions are already given.
    An execution of a YuchiKaml expression $e$, which has the history $h$ of built-in function calls and the return value $v$, is \emph{valid} if $(h,v)=\interp{e}$.
\end{definition}
