\subsection{Syntax}

\emph{Expressions} of YuchiKaml are defined by the following BNF equations:

\begin{align*}
    e \defeq    & () \mid x \mid n \mid \trueToken \mid \falseToken \mid s \mid (e)\\
                &\mid e\ e \mid \logNot e\\
                &\mid e * e \mid e / e\\
                &\mid e + e \mid e - e\\
                &\mid e \leq e \mid e < e \mid e \geq e \mid e > e\\
                &\mid e = e \mid e \not= e\\
                &\mid e \logAnd e\\
                &\mid e \logOr e\\
                &\mid e \triangleright e \mid e \composition e\\
                &\mid \ifToken e \thenToken e \elseToken e \mid \letToken (\recToken) x\ \tilde{a} = e \inToken e \mid \letToken \recToken x\ a_1\ \tilde{a} = e \inToken e\mid \lambda x \rightarrow e
\end{align*}

The operators defined in earlier rows have stronger precedences than the operators defined in later rows.
For example, $1 + 2 * 3$ is not parsed as $(1 + 2) * 3$, but $1 + ( 2 * 3)$.

In real source codes, the symbols above are notated as follows:

\begin{align*}
    &\leq & <=\\
    &\geq & >=\\
    &\not= & !=\\
    &\logAnd & \&\&\\
    &\logOr & ||\\
    &\triangleright & |>\\
    &\composition & >>\\
    &\lambda & \backslash\\
    &\rightarrow & ->
\end{align*}
