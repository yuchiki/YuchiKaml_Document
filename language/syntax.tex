
\subsection{Syntax}
In this section, we define the syntax of YuchikiML.

\subsubsection{Expression}

\emph{Expressions} of YuchiKaml are defined by the following BNF equations:

\begin{align*}
    e \defeq    & () \mid x \mid n \mid \trueToken \mid \falseToken \mid s \mid (e)\\
                &\mid e\ e \mid \logNot e\\
                &\mid e * e \mid e / e\\
                &\mid e + e \mid e - e\\
                &\mid e \leq e \mid e < e \mid e \geq e \mid e > e\\
                &\mid e = e \mid e \not= e\\
                &\mid e \logAnd e\\
                &\mid e \logOr e\\
                &\mid e \triangleright e \mid e \composition e\\
                &\mid \ifStructure{e}{e}{e} \mid \letStructure{x\ \tilde{a}}{e}{e} \mid \letRecStructure{f}{a_1\ \tilde{a}}{e}{e} \mid \abs{x}{e}\\
                &\mid e\opCol{}e
\end{align*}

The operators defined in earlier rows have stronger precedences than the operators defined in later rows.
For example, $1 + 2 * 3$ is not parsed as $(1 + 2) * 3$, but $1 + ( 2 * 3)$.

\begin{example}[GCD]
    This is an example of a YuchiKaml source code of a program which calculates the greatest common divisor of 120 and 45.

    \lstinputlisting[caption=gcd, language=ML]{samples/gcd}
\end{example}

\subsubsection{Syntax Sugar}
Some of the expressions shown above are syntactic sugars.
We show the list and syntax sugars and how they are desugared.

\begin{align*}
        e_1 \triangleright e_2 &\defeq e_2\ e_1 \\
        e_1 \composition e_2 &\defeq \abs{x}{g (f x)}\\
        \letStructure{f\ a_{1} \cdots a_n}{e_1}{e_2}&\defeq
            \letStructure{f}{\lambda a_1 . \cdots \lambda a_n.\ {e_1}}{e_2}\\
        \letRecStructure{f}{a\ b_{1} \cdots b_n}{e_1}{e_2}&\defeq
            \letRecStructure{f}{a}{\lambda b_1 . \cdots \lambda b_n.\ {e_1}}{e_2}\\
        e_1\opCol{}e_2 &\defeq \letStructure{\_}{e_1}{e_2}
\end{align*}

The semantics of expressions, introduced in a later section, is defined for desugared expressions.

\subsubsection{real notation}

In real source codes, the symbols above are notated as follows:

\begin{align*}
    &\leq & <=\\
    &\geq & >=\\
    &\not= & !=\\
    &\logAnd & \&\&\\
    &\logOr & ||\\
    &\triangleright & |>\\
    &\composition & >>\\
    &\abs{x}{e} & \backslash x -> e
\end{align*}
